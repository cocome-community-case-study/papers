\chapter{Design Details for Evolution Scenarios}
In this chapter we provide the detailed design documentation for each of the evolution scenarios
introduced in the prior section. Sec. \ref{App} sketches the design decision for the Mobile App that provides a second sales channel next to the existing Pick-up Shop. Sec. \ref{Docker} describes the adaptive changes of using a Docker environment to simplify the update process. They are both based on, or at least use the Hybrid Cloud-based Variant of CoCoME \cite{HeinrichRostamiReussner2016_1000052688}. In contrast, Sec. \ref{MS} provides a detailed design documentation of a new architectural version of CoCoME. This perfective evolution scenario is realized based on the Microservice idea.

\section{Adding a Mobile App Client} \label{App}
Developing the Mobile App Client as an extension of CoCoME requires additional use cases. They are described in Sec. \ref{UseCasesMobileApp}. Sec. \ref{DesignMobileApp} describes extensions on design level. The content of this chapter mainly originates from \cite{schnabel}.
\subsection{Use Cases of the Mobile App}\label{UseCasesMobileApp}
		\begin{figure}[t]
			\includegraphics[width=\textwidth]{img/appUseCase.png}
			\caption{Use Case Diagram CoCoME Mobile App}
		\end{figure}

\textbf{UC 14 - ProcessAppSale}\\
\textit{Brief Description} A Customer selects the product items s/he wants to buy and the payment by credit card is performed.\\ \newline
\textit{Involved Actors} AppCustomer, Bank\\ \newline
\textit{Precondition} The App is ready to process a new sale and the Customer already has an account registered in the System.\\ \newline
\textit{Trigger} The Customer opens the app and wants to buy product items.\\ \newline
\textit{Postcondition} The Customer has paid and the sale is registered in the inventory.\\ \newline
\textit{Standard Process}
\begin{itemize}[leftmargin=*]
	\item[1.] The AppCustomer searches products provided by the App.
	\item[2.] The AppCustomer can see details for each product on a separate site.
	\item[3.] The AppCustomer adds the product items s/he wants to purchase to the Shopping Cart. Step 1-3 is repeated until all items are added to the cart.
	\item[4.] The AppCustomer gets an overview of the items in the cart, their price and the running total.
	\item[5.] The AppCustomer proceed to the Checkout 
	\item[6.] The AppCustomer selects the Store where s/he wants to pick up his/her	purchased product items. 
	\item[7.] The AppCustomer is presented with a login form and is required to complete the "Authenticate user" use case.
	\item[8.] In order to initiate card payment, the customer selects a credit card used for the purchase.
	\item[9.] The AppCustomer enters his/her PIN in the designated field presented by the System.
	\item[10.] The System presents the Customer with an overview of the purchase, the AppCustomer confirms the purchase and waits for validation. Step 9 is repeated until the validation is successful or the Customer decides to cancel the purchase.
	\item[11.] Completed sales are logged by the System and sale information are sent to	the Inventory in order to update the stock.
	\item[12.] A success message is presented to the AppCustomer and the product items
	are being prepared to be picked up by the customer.
	\item[13.] The AppCustomer closes the app.
\end{itemize}

\textit{Alternative or Exceptional Processes}
\begin{itemize}
	\item[-] In step 8: No Card available
	\begin{itemize}
		\item[1.] In order to add a new credit card the Customer clicks the Add Card button.
		\item[2.] The Customer enters the card number of the new credit card and saves the card.
	\end{itemize}
	
	\item[-] In step 10: Card validation fails
	\begin{itemize}
		\item[1.] The Customer tries again and again.
		\item[2.] Otherwise, the Customer can decide to cancel the purchase.
	\end{itemize}
\end{itemize}


\textbf{UC 15 - CreateAppCustomer}\\ \newline
\textit{Brief Description} The app offers a possibility to create a new Customer account.\\ \newline
\textit{Involved Actors} AppCustomer\\ \newline
\textit{Precondition} The Customer does not have a Customer account yet and the app is started.\\ \newline
\textit{Trigger} A new AppCustomer wants to create an account.\\ \newline
\textit{Postcondition} The User is authenticated.\\ \newline
\textit{Standard Process}
\begin{itemize}[leftmargin=*]
	\item[1.] The AppCustomer has to fill out forms, requesting all necessary information to create a new AppCustomer account.
	\begin{itemize}
		\item[(a)] Form for name, email and password
		\item[(b)] Form for address
		\item[(c)] Summery of the information 
	\end{itemize}
	\item[2.]The Customer fills out the forms, verifies and submits the information.
	\item[3.] The app verifies the given information and creates a new Customer account in the Inventory.
\end{itemize}

\textit{Alternative or Exceptional Processes}
\begin{itemize}
	\item[-] In step 3 : Provided information is incorrect or not valid. The Customer is notified of the problem and enters the information again until it passes the check.	
\end{itemize}

\textbf{UC 16 - AuthenticateAppUser}\\ \newline
\textit{Brief Description} The app provides the possibility to authenticate a User.\\ \newline
\textit{Involved Actors} AppCustomer\\ \newline
\textit{Precondition} The app is started.\\ \newline
\textit{Trigger} An AppCustomer wants to authenticate his/herself.\\ \newline
\textit{Postcondition} The AppCustomer is authenticated.\\ \newline
\textit{Standard Process}
\begin{itemize}[leftmargin=*]
	\item[1.] The AppCustomer gets displayed a login form. S/he is asked to enter email and password.
	\item[2.] The App checks the provided credentials. If correct, the AppCustomer is logged in.
\end{itemize}

\textit{Alternative or Exceptional Processes}
\begin{itemize}
	\item[-] In step 2: Wrong credentials
	\begin{itemize}
		\item[1.] An error message is displayed.
		\item[2.] The User may try again until the authentication succeeds.
	\end{itemize}
\end{itemize}


\newpage

\subsection{Design of the Mobile App}\label{DesignMobileApp}
 Fig. \ref{ComponentApp} is the component of this evolution scenario. When adding the Mobile App client, the hybrid cloud-based variant of CoCoME did not have to be modified. Therefore, CoCoME is encapsulated in a single component. Simply the three web services \textit{WebService::Inventory::LoginManager, WebService::Inventory::Store and WebService::Inventory::Enterprise} used by the App Client are emphasized. The entire component diagram for the hybrid cloud-based variant is available in the Technical Report \cite{HeinrichRostamiReussner2016_1000052688}. 
 \\ Fig. \ref{ComponentApp} indicates that the AppShop requires an adapter to access the web services provided by CoCoME. This is because CoCoME uses SOAP/WS*-based web services which are not compatible with the technology used to implement the AppShop Client. A more detailed introduction about the technology used to implement the Mobile App Client can be found in \cite{schnabel}. 
  
 \begin{figure}[!h]
	\includegraphics[width=\textwidth]{img/appComponent.png}
	\caption{Component Diagram of the CoCoME Ecosystem After Adding the Mobile App Client}
	\label{ComponentApp}
\end{figure}
\noindent
  The \textit{AppShopAdapter} consumes the three web services \textit{WebService::Inventory::LoginManager, WebService::Inventory::Store} and \textit{WebService::Inventory::Enterprise} and provides a Rest Api which is used by the actual \textit{AppShop}. The Rest Api contains endpoints to retrieve and process Credit Card, Enterprise and StockItem information. To implement \emph{UC14-16}, the Api also provides endpoints for user management and processing sales. 
  
\begin{figure}[!h]
	\includegraphics[width=\textwidth]{img/appSearchSequence.png}
	\caption{Sequence Diagram of Searching an Item in the Mobile App Client}
	\label{SequenceAppSearch}
\end{figure}
\noindent
Fig. \ref{SequenceAppSearch} shows the process of opening a page to search for an Item. The customer opens the \textit{WebShopClient} and triggers the "Search" function to search for an item. To open the page, the \textit{NavigatorMenu} must call the \textit{Navigator} which creates a pagestate object and passes the object to the page. This HTML page is now presented to the customer. To fill the page with information, i.e when searching for a \textit{ProductItem}, the page uses services provided by the \textit{ServiceHolder}. In this case, the \textit{ItemService} calls the responsible REST-Service of \textit{AppShopAdapter} which in turn retrieves the necessary information from the WSDL services provided by CoCoME.

\noindent
Fig. \ref{SequenceAppSale} demonstrates how the Mobile App Client processes sales.  For the sake of clarity, the digram is simplified and only contains the most important calls. First, the customer searches for items (according to Fig. \ref{SequenceAppSearch}). By clicking on the desired Item, the according \textit{ItemPage} is shown. This page carries information about the Item. Here, the customer decides whether the Item should be added to the Shopping Cart or not. The last steps are repeated until the customer decides to proceed to the checkout. If not logged in, the customer gets forwarded to the \textit{LoginPage}. When successfully logged in, the customer clicks the \textit{BuyNow}-Button. The Sale process is finished as soon as the backend (CoCoME) has processed the sale.




\begin{figure}[!h]
	\includegraphics[width=\textwidth]{img/appProcessSale.png}
	\caption{Sequence Diagram of Processing a Sale}
	\label{SequenceAppSale}
\end{figure}

\newpage

\section{Using a Docker Environment} \label{Docker}

As shown in Fig. \ref{techStack}, using a Docker Environment affects the technology stack by adding additional layers. More detailed, the given CoCoME Stack is moved into the Docker Deamon, which runs a Linux distribution.The original parts of the stack, like Glassfish and the Java Virtual Machine, are still a part of the stack.
	
	\begin{figure}[!h]
		\centering
		\includegraphics[width = 0.5\textwidth]{img/tech_stack_CoCoME.png}
		\caption{Extended technology stack CoCoME}
		\label{techStack}
	\end{figure}
\noindent	
The Dockerfile defines an environment based on the latest version of Ubuntu 16:04. Maven, Git and Java are also installed using the standard Ubuntu package manager.\\
Git has two purposes: On the one hand it is used to download the most recent version of CoCoME.	On the other hand, it is used to download a prefabricated version of Glassfish that already includes domains and other adjustments required for CoCoME. Java is required by Glassfish and CoCoME as they need the Java Virtual Machine. Maven is needed to deploy the latest version of CoCoME onto the provided Glassfish servers.
\noindent
During the development, it was decided to implement and provide two different versions. The first version always pulls the most recent CoCoME source code from GitHub, downloads the entire dependencies with Maven, compiles and builds the project and finally deploys CoCoME on the Glassfish servers. As a consequence, creating and starting a Docker Container takes about one hour.\\
In contrast, the second version only pulls a prefabricated version of CoCoME from GitHub. 
Therefore, pulling the source code up to building the project is skipped. Maven does not have to be included in the technology stack. Solely, deploying CoCoME on the Glassfish server is necessary.\\
This reduces the deployment time to a few minutes but has a disadvantage: The prefabricated version is updated manually. Therefore, it is sometime not the most recent version.\\
By providing both, a fast deploying version and a current version, the user can choose what's the best for its situation.
	

	

	
\section{Microservices Technology} \label{MS}
	In this section we provide a brief design documentation of the use cases that are defined in the hybrid cloud-based variant of CoCoME \cite{herold2008}(p.4-10).
	The following subsections are divided into the Microservices and the corresponding Use Cases. Sec. \ref{archiOverviewMicro} describes the general architectural overview of the Microservice variant of CoCoME.
	

%TODO  Hier weiter
	

	\FloatBarrier
		\subsection{Orders}
		This section describes the design of the use cases implemented in the \textit{Orders} Microservice that provides main parts of the functionality for UC 3 and UC 4.\\
		\noindent
		\\
		\textbf{Behavioural View on UC 3 - Order Products} \\
		For better understanding, UC 3 is divided in two steps. The first part is described in Fig. \ref{MS_UC3_1}: A user chooses the product items to order and the corresponding amount. Each \textit{ProductOrder} is stored in the \textit{ProductOrderRepository} as a \textit{OrderEntry}. In the second part (Fig. \ref{MS_UC3_2}), the  \textit{OrderEntries} are wrapped in a single \textit{ProductOrder} element that additionally contains information about the store and the date of the order. When the user presses the button \textit{Order}, the \textit{OrderManagement} iterates over the collection of \textit{OrderEntries} and sets a reference to the actual \textit{ProductOrder}.
	
		\noindent
		\\
		\textbf{Behavioural View on UC 4 - Receive Ordered Products} \\
		//%TODO ÜBERARBEITEN: Das ergibt keinen sinn. Der UC hat laut dem Buchkapitel irgendwie einene anderen Sinn. Kommen da einzelne Produkte an, oder kommt da die ganze Order an? Worüber wird iteriert usw.     Vergleich es mal mit dem UC4 Diagramm aus dem Buch
		The figure \ref{MS_UC4} shows, that first the ProductOrder element is refreshed and the delivery date is set to the passed date.
		After this, it performs for each OrderEntry a rest call to the store microservice in which the store itself is represented. With that rest call, the StockItem, which is representing the corresponding product in the store ,is increased the amount of available products.
		
			\begin{figure}[!h]
				\centering
				\includegraphics[width = 0.7\textwidth]{img/UC3_Order_Products_1_OrderEntryCreate.jpg}
				\caption{Usecase 3 order products, part 1}
				\label{MS_UC3_1}
			\end{figure}
			
			\begin{figure}[!h]
				\centering
				\includegraphics[width = 1\textwidth]{img/UC3_Order_Products_2_ProductOrderCreate.jpg}
				\caption{Usecase 3 order products, part 2}
				\label{MS_UC3_2}
			\end{figure}
		
			
			\begin{figure}[!h]
				\centering
				\includegraphics[width = 1\textwidth]{img/UC4_Receive_Ordered_Products.jpg}
				\caption{Usecase 4 receive ordered products}
				\label{MS_UC4}
			\end{figure}
			
		\FloatBarrier
			
		\subsection{Stores}
			This section describes the design of the use cases implemented in the \textit{Store} Microservice that provides main parts of the functionality for UC 1, UC 2, UC 7 and UC 8.\\
	
		\noindent
		\\
		\textbf{Behavioural View on UC 1 - Process Sale} \\
		Again UC 1 is divided in two parts (Fig. \ref{MS_UC1_1} and Fig. \ref{MS_UC1_2}).
		The first part describes how a user can add products to the sale process. When the \textit{startSale} action is executed, the sale mode is activated and the \textit{CashDesk} is resetted from a previous, probably cancelled sale processes. To add products to the sale process, the user can either enter the barcode digit by digit using the keyboard or scan the barcode using a \textit{Scanner}. Further, the user can choose how many products s/he wants to purchase. This process is depicted in the inner loop. 
		\\
		Several checks are executed when successfully entering the barcode: If an item with the same barcode was already added to the sale, then the total amount for this item is increased by the second purchase amount. If not, the item is added to the sale, provided that an item with this barcode exists. In both cases, the availability of the item amount in the stock is checked and reduced. If one of the conditions is violated, the attempt of adding a product with the provided barcode an amount is quit. 
		Subsequently, the display is updated and the product information is added to the printer output.
		\\
		The second part of this use case, shown in Fig. \ref{MS_UC1_2}, handles the end of the sale process. By calling the \textit{finishSale} routine when pressing the button \textit{FinishSale}, the display is updated. Now, the user needs to choose between paying by card or cash and the \textit{CashDesk} is set to the corresponding paying mode. In case the user wants to pay by credit card, s/he needs to enter the credit card details. In the other case, the cash amount is entered. In both cases, the information is checked to accuracy.
		After successfully ending the payment, the printer and display are updated and the \textit{CashDesk} ends the sale process.

		\noindent
		\\
		\textbf{Behavioral View on UC 2 - Manage Express Checkout} \\
		Figure \ref{MS_UC2} shows, how each cash desk performs on an external call the updateExpressLight routine. In this, the cash desk checks by itself if based on the recently proceeded sales the Express light should be activated or not. In consequence, the express light is set to the correct value. 
		
		\noindent
		\\
		\textbf{Behavioural View on UC 7 - Chane Price} \\
		As described in figure \ref{MS_UC7}, in a first step, the database element which represents the store, managed by the store manager who wants to change the price is loaded. On this, the available products are filtered to find the one with the correct productId. 
		Having the correct StockItem found, the sale price of that object can be updated. Afterwards, the StockItem object is updated in the database.
		
		\noindent
		\\
		\textbf{Behavioural View on UC 8 - Product Exchange} \\
		The process on usecase 8 is shown in figure \ref{MS_UC8}. It starts at the end of an sale process. The microservice checks by itself if the stock amount of the sold items have reached their minimal amount. If not, nothing will happen. Otherwise, this information is passed to the appropriate EnterpriseManager. 
		This EnterpriseManager collects in a first step all of its stores, which are also selling this specific product.
		If the EnterpriseManage has found at least one store which also provides that products, it figures out, which one of the possible exchange sources is the best. By reducing the stock amount of that product in this optimal store and increasing it in the store, which started the request, the product exchange is executed.
		
			\begin{figure}[!h]
				\centering
				\includegraphics[width = 1\textwidth]{img/UC1_Process_Sale_1_StartSaleAndProductChoosing.jpg}
				\caption{Usecase 1 process sale, part 1}
				\label{MS_UC1_1}
			\end{figure}
			
			\begin{figure}[!h]
				\centering
				\includegraphics[width = 1\textwidth]{img/UC1_Process_Sale_2_FinishSaleAndPayment.jpg}
				\caption{Usecase 1 process sale, part 2}
				\label{MS_UC1_2}
			\end{figure}
			
			\begin{figure}[!h]
				\centering
				\includegraphics[width = 0.5\textwidth]{img/UC2_Manage_Express_Checkout.jpg}
				\caption{Usecase 2 manage express mode}
				\label{MS_UC2}
			\end{figure}
			
			\begin{figure}[!h]
				\centering
				\includegraphics[width = 0.9\textwidth]{img/UC7_Change_Price.jpg}
				\caption{Usecase 7 change price}
				\label{MS_UC7}
			\end{figure}
			
			\begin{figure}[!h]
				\centering
				\includegraphics[width = 1\textwidth]{img/UC8_Product_Exchange.jpg}
				\caption{Usecase 8 product exchange}
				\label{MS_UC8}
			\end{figure}
		\FloatBarrier	
		
		\subsection{Reports}
		This section describes the design of the \textit{Reports} microsevice, which provides the functionality of the usecases 5 and 6.
		
		 \noindent
		 \\
		\textbf{Behavioural View on UC 5 and Uc 6 - Show Stock Report and Show Delivery Report} \\
		In the following, the realization of usecase 5 and 6 are described. Usecase 5 is described in figure \ref{MS_UC5} and usecase 6 in figure \ref{MS_UC6}. 
		In both cases, this microservice performs an rest call to receive informations from the microservice which contains the requested informations. This informations are returned on that request and are modified to represent the information in an proper format. That format is passed back to the calling frontend.
			
			\begin{figure}[!h]
				\centering
				\includegraphics[width = 1\textwidth]{img/UC5_Show_Stock_Report.jpg}
				\caption{Usecase 5 show stock report}
				\label{MS_UC5}
			\end{figure}
			
			\begin{figure}[!h]
				\centering
				\includegraphics[width = 1\textwidth]{img/UC6_Show_Delivery_Reports.jpg}
				\caption{Usecase 6 show delivery report}
				\label{MS_UC6}
			\end{figure}
			
	\FloatBarrier
	\subsection{Architectural overview}	\label{archiOverviewMicro}	
	This section describes the architectural design of the microservices. As can be seen in figure \ref{MS_ARch}, there are the four microservices \textit{Orders}, \textit{Reports} \textit{stores} and \textit{Produkts}. The microservices provides its own user interface pages, which can be loaded dynamically. \\
	Each of the microservices provides functionality which primary works on the database, located in that microservice. More detailed, these are for the Orders microservice functionality like creating new orders, delivering orders and showing those. The reports microservice does not possess an own database. Instead this one collects informations by rest request to the other microservices. The stores service provides functionality, needed by enterprisemanagers or storemanagers for example creating new stores or changing the sales prices for goods. Further, this one also contains the functionality for the sale process. In detail, the cash desk with its accessories is implemented here. In microservice products functionality for creating new supplier or products is provided, but also for ordering them.
	
	
	\begin{sidewaysfigure}[ht]
	   	\includegraphics[width=\textwidth]{img/MicroserviceArchitecture.jpg}
	   	\caption{Microservice architecture}
	   	\label{MS_ARch}
	\end{sidewaysfigure}
